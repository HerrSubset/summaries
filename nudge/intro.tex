\section{Introduction}

T\&S\footnote{Thaler and Sunstein, the authors.} start the book with the cafetaria exmaple. Using this example, they coin the term \emph{choice architect}. This is someone who is responsible ``for organizing the context in which people make decisions''. Just like with other types of architecture, little things can have great impact, and no design is neutral.

Because of the possibly great effect on people's lives, T\&S propose their idea/movement of \emph{libertarian paternalism}. 
\begin{description}
    \item[libertarian:] liberty-preserving, choice should not be restricted, people should be able to go their own way
    \item[paternalism:] choice architects can and should try to influence people into making their lives longer/healthier/better (they should \emph{nudge} them)
\end{description}
There is a simple reason for nudging being legitimate: people, in general, are pretty shit at making good decisions. Unlike traditional economics assumes, real people (humans) are not as perfect as the ``homo economicus''(econs).

\begin{description}
    \item[nudge:] ``any aspect of the choice architecture that alters people's behavior in a predictable way without forbidding any options or significantly changing their economic incentives.''
\end{description}

People opposing paternalism argue that human beings make good decisions. Better than anyone would do making the decision for them. These people often think that maximizing the number and variety of choices is the best way to go. The surge in obesity rates points out that this thinking is flawed. Besides this false assumption, there are two misconceptions surrounding paternalism:

\begin{enumerate}
    \item It is possible to avoid influencing people's choices. Research shows how strong inertia/default options can be.
    \item Paternalism always involves coercion. Not true as the cafetaria example shows that choice architecture can preserve the entire offering. It does not force a diet on anyone.
\end{enumerate}
